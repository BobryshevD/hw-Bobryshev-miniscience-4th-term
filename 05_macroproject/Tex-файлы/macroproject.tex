%%
%% ****** ljmsamp.tex 13.06.2018 ******
%%
\documentclass[
11pt,%
tightenlines,%
twoside,%
onecolumn,%
nofloats,%
nobibnotes,%
nofootinbib,%
superscriptaddress,%
noshowpacs,%
centertags]%
{revtex4}
\usepackage{ljm}
% \usepackage{cmap}					% поиск в PDF
% \usepackage{mathtext} 				% русские буквы в формулах
% \usepackage[T2A]{fontenc}			% кодировка
%\usepackage[utf8x]{inputenc}			% кодировка исходного текста
%\usepackage[russian]{babel}	% локализация и переносы

\begin{document}

\titlerunning{Численное решение волнового уравнения} % for running heads
\authorrunning{Бобрышев Д.Ю.} % for running heads
%\authorrunning{First-Author, Second-Author} % for running heads

\title{Численное решение волнового уравнения}
% Splitting into lines is performed by the command \\
% The title is written in accordance with the rules of capitalization.

\author{\firstname{Д.~Ю.}~\surname{Бобрышев}}
\email[E-mail: ]{bobryshev.diu@phystech.edu}
%\affiliation{Place of work and/or the address of the first and second authors}
\affiliation{Московский физико-технический институт, Институтский пер., 9, Долгопрудный, Московская обл., 141701}

\author{\firstname{С.~М.~М.}~\surname{Аль-Хадж Аюб}}
\email[E-mail: ]{al-khadzh.aiub.sm@phystech.edu}
%\affiliation{Place of work and/or the address of the first and second authors}
%\affiliation{Place of work and/or the address of second authors}
%\noaffiliation % If the author does not specify a place of work.

%\firstcollaboration{(Submitted by Я)} % Add if you know submitter.
%\lastcollaboration{ }

\received{22 Мая, 2024} % The date of receipt to the editor, i.e. December 06, 2017


\begin{abstract} % You shouldn't use formulas and citations in the abstract.
Основной целью данной работы является наблюдение различных эффектов, возникающих при решении волнового
уравнения. Численное решение выполнено при помощи метода конечных элементов, написано на языке C++ 
с использованием библиотеки FEniCS. 


\end{abstract}

%\subclass{12345, 54321} % Enter 2010 Mathematics Subject Classification.

\keywords{Волновое уравнение, FEniCS} % Include keywords separeted by comma.

\maketitle

% Text of article starts here.

\section{Введение}
Волновое уравнение (1) является одним из основных уравнений математической физики. При помощи
волнового уравнения описываются различные колебательные процессы, например, 
распространение звуковой волны в среде, распространение электромагнитных волн. В данной работе
мы будем использовать теорминологию, используемую в электродинамике, а полученное решение 
следует интерпретировать как распространение электромагнитной волны. \newline
\begin{equation}
    \Delta u = \frac{1}{v^2}\frac{\partial^2u}{\partial t^2}
\end{equation}
При решении волнового уравнения возникает ряд интересных эффектов. Например, выполняется принцип
Гюйгенса-Френеля: каждая точка волнового фронта является источником вторичных
когерентных сферических волн. При некоторых условиях, как следствие этого приниципа, 
имеет место дифракция: явление огибания предметов световой волной. Так, например, если осветить
узкую щель плоской монохроматической волной, на достаточно удалённом экране можно наблюдать
последовательность светлых полос. Наблюдение дифракции является одной из основных целей
работы.\newline
Из принципа Гюйгенса-Френеля следует закон преломления волны при прохождении границы двух сред
с различными показателями преломления. Преломлённая волна имеет иную скорость распространения, поэтому
у преломлённой волны угол между направлением распространения и нормалью к поверхности оказывается
иным, чем угол между нормалью и направлением распространения падающей волны. Данное явление
можно использовать, например, для фокусировки волны в точку. Для этого используются
линзы. Исследование свойств линз также является важной частью работы. \newline
Для решения волнового уравнения мы использовали метод конечных элементов. Основной идеей данного
метода является разбиение пространства, на котором необходимо получить решение, на некоторое
конечное количество элементов: точек со связывающих их отрезками. Для каждой пары отрезков, имеющих
общую точку, можно ввести линейную функцию $v_k$, и число $u_k$, а саму функцию аппроксимировать как
сумму базисных функций $v_k$, умноженных на коэффициент $u_k$. Таким образом, дифференциальное уравнение
с некоторыми граничными условиями можно приближённо представить в виде системы линейных уравнений,
которую можно решить относительно $u_k$, и, таким образом, получить приближённо функцию u. Метод 
конечных элементов реализован в библиотеке FEniCS, которую мы использовали для получения численного
решения волнового уравнения.


\section{Теоретическая часть}



\end{document}
